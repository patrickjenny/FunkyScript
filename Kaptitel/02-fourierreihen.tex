\section{Fourierreihen}
	Temperaturverteilung auf Ring
	$$a_0 + \sum_0^\infty (a_n \cos(n\omega\phi) + b_n \sin(n\omega\phi))$$

	P.L. Dirichlet (1829) $\rightarrow$ mathematischer Beweis

	Zerlegung einer periodischen Funktion nach diskreten Teilfrequenzen
	\begin{itemize}
		\item Fourieranalyse
		\item Fouriersynthese
	\end{itemize}

	Periodische Funktion mit Periodenlänge $L$
	\begin{figure}[H]
		\centering
		\includegraphics[width=0.7\linewidth]{Grafiken/2_Fourierreihen/Grafik1.PNG}
		\caption{Periodische Funktion mit Periodenlänge $L$}
		\label{}
	\end{figure}

	\begin{Def}
		Eine Funktion $f: \mathbb{R}\rightarrow\mathbb{R}(\mathbb{C})$ wird periodischen
		mit Periode $L$ , $L>0$, genannt wenn:
		$$f(x+L) = f(x) \quad \forall x \in \mathbb{R}$$
	\end{Def}

	\begin{Bem}
		Periodiesche Funktion mit Periode $L$ ist eindeutig auf ganz $\mathbb{R}$ fesgelegt,
		wenn man sie auf einem beliebiges Intervall der Länge $L$, $[a,a+L)$ kennt.\\
		Standardvorgabe:
		$$[0,L) \quad \textrm{oder} \quad [-\frac{L}{2}, \frac{L}{2})$$
	\end{Bem}

	\begin{Bem}
		$f(x)$ periodisch mit Periode $L$
		$$\Rightarrow \int_{-\frac{L}{2}}^{\frac{L}{2}} \,dx f(x) = \int_{a-\frac{L}{2}}^{a+\frac{L}{2}} \,dx f(x)$$
	\end{Bem}

	Betrachte Funktion $f: \mathbb{R}\rightarrow\mathbb{R}(\mathbb{C})$ mit
	\begin{enumerate}[label=(\roman*)]
		\item periodisch mit Periode L, d.h. $f(x+L) = f(x) \quad \forall x \in \mathbb{R}$
		\item Lebesque-integrierbar auf dem Periodizitätsintervall, dh $f \in L^1(-\frac{L}{2},\frac{L}{2})$ (schwächer als quadratintegrierbar)
	\end{enumerate}

	Betrachte Funktionen der Einfachheit halber für $L=2\pi$\\
	Die $\infty$-trigometrische Reihe
	$$FR(f)(x)= \frac{1}{2} a_0 + \sum_{n=1}^\infty (a_n \cos(n x) + b_n \sin(n x)) \quad \textrm{mit}$$
	$$a_n = \frac{1}{\pi} \int_{-\pi}^{\pi} \,dx f(x)\cos(nx) \quad n \geq 0 \quad n \in \mathbb{N}$$
	$$b_n = \frac{1}{\pi} \int_{-\pi}^{\pi} \,dx f(x)\sin(nx) \quad n>0 \quad n \in \mathbb{N}$$
	heißt Fourierreihe der Funktion $f$ mit den Fourierkoeffizienten $a_n$ und $b_n$.

	Konvergenz der Partialsummen

	$FR(f)(x)= \frac{1}{2} a_0 + \sum_{n=1}^N (a_n \cos(n x) + b_n \sin(n x))$

	muss beachtet werden.

	\begin{Bem}
		Wenn $f \in L^1(-\pi,\pi)$, dann stellt Fourierreihe eine Entwicklung nach der
		OGB

		$\{1, \sin(nx), \cos(nx); n \in \mathbb{N}\}$ auf $\sqrt{2}$ normiert $\Rightarrow \frac{1}{2}$ bei $a_0$

		bzgl. des Skalarprodukts

		$$\langle f,g \rangle =  \frac{1}{\pi} \int_{-\pi}^{\pi} \,dx \overline{f(x)} g(x)$$

		dar.

		$\Rightarrow FR(f)$ konvergiert in $L^2$-Norm (Konvergenz im Mittel)

		$$\lim_{n\rightarrow \infty} {\Vert f- FR_N(t) \Vert}_2 = 0$$
	\end{Bem}

	\begin{Satz}{Parsewal I}
		$$\Vert f \Vert^2 = \frac{1}{2} \vert a_0 \vert^2 + \sum_{n=1}^{\infty} \left(\vert a_n \vert^2 + \vert b_n \vert^2\right)$$
	\end{Satz}

	\subsubsection{Rechenregeln und Beispiele}
	\begin{enumerate}[label=(\roman*)]
		\item Allgemeines Periodizitätsintervall $(a,b)$
			$$FR(f)(x)= \frac{1}{2} a_0 + \sum_{n=1}^\infty \left( a_n \cos\left(\frac{2\pi n x}{b-a}\right) + b_n \sin\left(\frac{2\pi n x}{b-a}\right) \right)$$
			$$a_n = \frac{2}{b-a} \int_{a}^{b} \,dx f(x) \cos \left(\frac{2\pi n x}{b-a}\right) \quad n \geq 0$$
			$$b_n = \frac{2}{b-a} \int_{a}^{b} \,dx f(x) \sin \left(\frac{2\pi n x}{b-a}\right) \quad n>0$$	

		\item Integrale über ein symetrisches Integrationsintervall 
			$\left(-\frac{L}{2}, \frac{L}{2} \right)$ um Null verschwinden
			für ungerade Integranden:
			\begin{itemize}
				\item $f(x) \quad \textrm{ungerade} \Leftrightarrow f(x) = -f(-x)$ 
					$$\frac{2}{L} \int_{-\frac{L}{2}}^{\frac{L}{2}} \,dx \underbrace{\underbrace{f(x)}_{ungerade} \underbrace{\cos\left(\frac{2\pi}{L} n x\right)}_{gerade}}_{ungerade} = 0 \quad \cos()\textrm{-Terme verschwinden}$$
				\item $f(x) \quad \textrm{gerade} \Leftrightarrow f(x) = f(-x)$
					$$\frac{2}{L} \int_{-\frac{L}{2}}^{\frac{L}{2}} \,dx \underbrace{\underbrace{f(x)}_{gerade} \underbrace{\sin\left(\frac{2\pi}{L} n x\right)}_{ungerade}}_{ungerade} = 0 \quad \sin()\textrm{-Terme verschwinden}$$
			\end{itemize}
		\item Integrale über ein Vielfaches der Periode von $\sin$- oder 			$\cos$-Funktion verschwinden
			$$\int_{a}^{a+k\frac{L}{n}} \sin\left(\frac{2\pi}{L} n x\right) \,dx =
			\int_{a}^{a+k\frac{L}{n}} \cos\left(\frac{2\pi}{L} n x\right) \,dx = 0
			\quad k\neq 0 \quad n \neq 0$$
			\begin{figure}[H]
				\centering
				\includegraphics[width=0.4 \linewidth]{Grafiken/2_Fourierreihen/Grafik2.PNG}
			\end{figure}
		\item Linearität der Fourierreihe
			$$FR(f+g)(x) = FR(f)(x) + FR(g)(x)$$
			$$FR(\alpha \cdot f)(x) = \alpha \cdot FR(f)(x)$$
	\end{enumerate}
	
	\begin{Bsp}{Sägezahnfunktion}
		$$f(x)=x \quad x \in (-\pi,\pi) \quad \textrm{Periode: } 2\pi$$
		\begin{figure}[H]
			\centering
			\includegraphics[width=0.4\linewidth]{Grafiken/2_Fourierreihen/Grafik3.PNG}
		\end{figure}

		$$a_{n \geq 0} = \frac{1}{\pi} \int_{-\pi}^{\pi} \,dx x \cos(nx) = 0$$

	\end{Bsp}